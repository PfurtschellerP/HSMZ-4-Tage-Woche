\chapter{Durchführung der Umfrage und der Datenaufbereitung \textit{\textcolor{gray}{(Autoren: Philipp und Patrick)}}}

\section{Durchführung}

Nach der Erstellung des Forschungsplans und des darin bereits beschriebenen Vorgehensweise
wurden die Fragen für die Umfrage auf Basis der in Tabelle \ref{tab:hyptothesen} aufgeführten Hypothesen 
ausgearbeitet. Die anschließende praktische Umsetzung erfolgte mithilfe der Software LimeSurvey 
\parencite[vgl.][]{limesurvey_gmbh_limesurvey_nodate}. 

% Nach einer Einführung in die Thematik der 4-Tage-Woche und einer sorgfältigen Auswahl der zu 
% überprüfenden Hypothesen (vgl. \ref{tab:hyptothesen}) wurde ein Fragebogen erstellt.
% Die anschließende praktische Umsetzung erfolgte mithilfe der Software LimeSurvey 
% \parencite[vgl.][]{limesurvey_gmbh_limesurvey_nodate}. Der vollständige Fragebogen befindet sich im Anhang \ref{app:umfrage}.

\begin{table}[h]
    \centering
      \begin{tabular}{|l|p{0.8\textwidth}|}
      \hline
      \multicolumn{2}{|c|}{Einstellung zur 4-Tage-Woche} \\ \hline
      H1 & Je jünger die Person, desto positiver steht sie der 4-Tage-Woche gegenüber. \\ \hline
      H2 & Je höher die Stellung eines Mitarbeiters im Unternehmen, desto weniger wird die 4-Tage-Woche als positiv angesehen. \\ \hline
      H3 & Die 4-Tage-Woche ist bei Eltern attraktiver als bei Kinderlosen. \\ \hline
      H4 & Die 4-Tage-Woche wird von einer Mehrheit der Befragten als zukünftig relevantes Thema bewertet. \\ \hline
      \multicolumn{2}{|c|}{Auswirkungen auf Arbeitgeber} \\ \hline
      H5 & Die Einführung einer 4-Tage-Woche führt zu einer Steigerung der Attraktivität des Arbeitgebers. \\ \hline
      H6 & Die Einführung der 4-Tage-Woche führt zu weniger Fluktuation. \\ \hline
      \multicolumn{2}{|c|}{Auswirkung auf Arbeitnehmer} \\ \hline
      H7 & Durch die Einführung der 4-Tage-Woche wird sich die Anzahl der Krankheitstage reduzieren. \\ \hline
      H8 & Durch die Einführung der 4-Tage-Woche kommt es zu weniger verfrühten Renteneintritten. \\ \hline
      H9 & Durch die Einführung der 4-Tage-Woche steigt die Bereitschaft zur Weiterbildung. \\ \hline
      \end{tabular}
      \caption{Hypothesen}
      \label{tab:hyptothesen}
\end{table}

Die Umfrage wurde im Zeitraum vom 04.05.2024 bis zum 29.05.2024 durchgeführt und es beteiligten sich
insgesamt 344 Personen an der Umfrage. Der vollständige Fragebogen befindet 
sich im Anhang \ref{app:umfrage}.

\section{Datenaufbereitung}
Von den 344 Personen, die sich beteiligt haben haben 282 Personen die Umfrage vollständig beantwortet.
62 Personen haben die Umfrage entweder nicht vollständig beantwortet oder bei Frage QA5 \gqq{Sind sie berufstätig?} 
mit \gqq{nein} geantwortet. Da diese Umfrage darauf abzieht Informationen über die 
Einstellung von Beschäftigten zu der 4-Tage-Woche zu erhalten, werden auch nur die 
Antworten von Personen, die sich in einem Beschäftigungsverhältnis befinden berücksichtigt. 
Somit konnten die Daten von 282 Personen für die Auswertung herangezogen werden.

Außerdem wurde bei Frage QA12, welche nach der Branche fragt in der die Person tätig ist,
bei Antworten mit \gqq{Sonstiges}, die Freitextantworten den bestehenden Kategorien zugeordnet,
bei denen sich die angegeben Antworten eindeutig einer Antwortmöglichkeiten zuordnen ließen. 
% Freitextantworten wie beispielsweise \gqq{Einzelhandel} dem Bereich \gqq{Handel} 
% und die Antwort \gqq{Beratung} dem Bereich \gqq{Dienstleistung} zugeordnet.

Zusätzlich zu den inhaltlichen Anpassungen wurden auch die Daten selbst für \ac{SPSS} 
% \parencite[vgl.][]{ibm_spss_nodate} 
aufbereitet. Dabei wurden die Spalten mit den Daten entsprechen mit einem aussagekräftigen kurzem
Variablennamen versehen, um die Verarbeitung der Daten in \ac{SPSS} zu erleichtern. Zusätzlich wurden 
den ordinalen und nominalen Daten nummerische Werte zugewiesen.
% , um die Daten in \ac{SPSS} korrekt verarbeiten zu können.
In \ac{SPSS} selbst wurden dann noch die Ergebnisse gelabelt, so dass ersichtlich ist, welche Antworten hinter den 
nummerischen Werten stehen.

% Zusätzlich zu der Anpassung der Daten wurde eine angepasste CSV-Datei für den Import 
% in \ac{SPSS} \parencite[vgl.][]{ibm_spss_nodate} erstellt. Hierbei wurden den Fragen Variablennamen 
% zugeordnet, sowie ordinalen und nominalen Daten nummerische Werte zugewiesen. Eine entsprechende 
% Aufführung davon findet sich in der unten aufgeführten Tabelle (vgl. \ref{tab:allgemeine_daten}). 