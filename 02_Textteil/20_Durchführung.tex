\chapter{Durchführung der Umfrage und der Datenaufbereitung}

%Was auch immer Gruppe 3 gemacht hat und wie die Daten aufbereitet wurden (Ausschließen von 
%nicht vollständigen Antworten, etc.)

% Online-Umfrage mit LimeSurvey 
% Zeitraum: 04.05.2024 – 29.05.2024 
 
% Antworten  
% Vollständige Antworten: 282 
% Unvollständige Antworten: 62 
% Antworten gesamt: 344 
\section{Durchführung}

Nach einer Einführung in die Thematik der 4-Tage-Woche und einer sorgfältigen Auswahl der zu 
überprüfenden Hypothesen wurde ein Fragebogen erstellt.
Die anschließende praktische Umsetzung erfolgte mithilfe der Software LimeSurvey 
\parencite[vgl.][]{limesurvey_gmbh_limesurvey_nodate}.

Die Umfrage wurde im Zeitraum vom 04.05.2024 bis zum 29.05.2024 durchgeführt und es beteiligten sich
insgesamt 344 Personen an der Umfrage. Davon haben 282 Personen die Umfrage vollständig beantwortet,
während 62 Personen die Umfrage nicht vollständig beantwortet haben. Somit konnten die Daten von 282 Personen für die Auswertung herangezogen werden.

\section{Datenaufbereitung}
Aus den Ergebnissen wurden für diese Umfrage nicht relevante Ergebnisse entfernt. 
Hierbei handelte es sich um 3 Datensätze, bei denen die Frage QA5 “Sind sie berufstätig?” 
mit “nein” beantwortet wurde. Da diese Umfrage darauf abzieht Informationen über die 
Einstellung von Beschäftigten zu der 4-Tage-Woche zu erhalten, werden auch nur die 
Antworten von Personen, die sich in einem Beschäftigungsverhältnis befinden berücksichtigt. 
Außerdem wurde bei Frage QA12 other in den Fällen, bei denen sich die angegeben Antworten 
eindeutig einem Bereich zuordnen ließen, dies getan. So wurde beispielsweise die Antwort 
“Einzelhandel” dem Bereich “Handel” zugeordnet und die Antwort “Beratung” dem Bereich 
“Dienstleistung” zugeordnet.

Zusätzlich zu der Anpassung der Daten wurde eine angepasste CSV-Datei für den Import 
in \ac{SPSS} \parencite[vgl.][]{ibm_spss_nodate} erstellt. Hierbei wurden den Fragen Variablennamen zugeordnet und ordinalen 
und nominalen Daten nummerische Werte zugewiesen. Eine entsprechende Aufführung davon 
findet sich in der unten aufgeführten Tabelle (vgl. \ref{tab:allgemeine_daten}). 