%Hypothese 9: Durch die Einführung der 4-Tage Woche steigt die Bereitschaft zur Weiterbildung.
%V&L

\chapter{Überprüfung der Hypothese 9}
\label{chap:hypothese9}

\section{Vorgehensweise}
Mit der Hypothese \gqq{Durch die Einführung der 4-Tage Woche steigt die Bereitschaft 
zur Weiterbildung} wird ein Fokus auf die 
Auswirkung der 4-Tage-Woche auf die Arbeitnehmenden gelegt. Auf Basis dieser Hypothese 
wurde in der Umfrage
insbesondere die Frage \gqq{Würde eine 4-Tage-Woche Ihre Bereitschaft zur Weiterbildung in 
Ihrer Freizeit erhöhen?} an die Teilnehmenden gestellt.

\paragraph*{Identifikation der relevanten Variablen}

Zu Beginn werden die unabhängigen, sowie die abhängige Variable identifiziert, welche für die Betrachung der
Hypothese 9 relevant sind.

Die abhängige Variable beinhaltet die Daten zu der Antwort auf die Frage aus der Umfrage
\gqq{Würde eine 4-Tage-Woche Ihre Bereitschaft zur Weiterbildung in Ihrer Freizeit erhöhen?}.
Sie spiegelt direkt die Bereitschaft der Teilnehmenden zur Weiterbildung in ihrer Freizeit 
unter einer 4-Tage-Woche wider.

Für die Unabhängigen Variablen wurden die folgenden Variablen aus der Umfrage ausgewählt. 
Hier wird angenommen, dass diese Variablen einen Einfluss auf die Bereitschaft zur 
Weiterbildung der Teilnehmenden haben könnten.
\begin{itemize}
    \item Alter
    \item Kinder (Nominal)
    \item Vollzeit/Teilzeit
    % \item Führungsverantwortung


\paragraph*{Ausgewählte Analyseverfahren}

Für die Auswertung aller abgegebenen Antworten wird zuerst ein Histogramm erstellt, um die gesamte Meinung
aller Teilnehmenden ohne Berücksichtigung der unabhängigen Variablen zu betrachten.
Hierzu werden ebenfalls der Modalwert (Modus), Median und Mittelwert der Antworten berechnet, um weitere
Fakten über die Verteilung der Antworten zu erhalten.

Im Anschluss werden die bereits genannten unabhängigen Variablen durch Kreuztabellen betrachtet, 
welche einen Einfluss auf die Bereitschaft zur Weiterbildung in der Freizeit unter einer 4-Tage-Woche 
haben könnten. 
Für eine genauere Analyse wird eine Regressionsanalyse durchgeführt, 
um einen tieferen Einblick in die Zusammenhänge zu erhalten. % Eventuell rausnehmen, wenn Kapitel zu lang wird





% Es werden Strukturprüfende Methoden verwendet, wenn man schon eine Vermutung hat
% (Regressionsanalyse S08, Varianzanalyse, Diskriminanzanalyse S11, Kontingenzanalyse)
% Regression geht nur bei allen Daten (?)

% Betrachtung: Ist die Bereitschaft zur Weiterbildung im allgemeinen da
% und besteht ein Zusammenhang zwischen der Bereitschaft zur Weiterbildung und bestimmten Gruppen



% Schauen ob bestimmte Gruppen von Leuten eher bereit sind für eine Weiterbildung
    % Alter - Zusammenhang: Je jünger, desto eher bereit?
    % Kinder (Nominal) - Einfache Regression?
    % Ausschließen von Teilzeit oder nicht?
    % Führungsverantwortung - Sind die Personen schon gebildet genug?

    % Auf Statistische Unabhängigkeit achten/prüfen?




\section{Analyse}
% Histogramm mit Auswertung der 

% Modus, Mittelwert und Median der Antworten anschauen
% Anhand der Lage von Modus, Median und Mittelwert zueinander lässt sich erkennen wie die 
% Häufigkeitsverteilung aussieht. Dabei stellte sich heraus, dass die Antworten ...
% Dies bedeutet, dass ...

% Themen wie Geschlecht oder Alter hier reinpacken?

% Korrelationskoeffizienten anschauen? Nur wenn Korrelationen zwischen den Variablen da ist,
% dann auch Regressionsanalyse durchführbar?

% 08 Teil 3

% 08 Teil 4
    % 1. Modellbildung - Notwenig? Nö, wegen SPSS?
    % 2. Berechnung der Regressionsfunktion - 2.5 (S8-97) Schrittweise?
    % 3. Test der Regressionsfunktion - Güte des Modells
    % 4. Test der Regressionskoeffizienten
    % 5. Test der Modellprämissen

\section{Ergebnis}

% Beantworten, ob die Hypothese abgelehnt oder beibehalten wird

% Annahme, da die Umfrage in dem näheren Umfeld von Personen durchgeführt wurde, die 
% Berufsbegleitend einen Master machen, kann vermutet werden, dass die Bereitschaft zur
% Weiterbildung in der Freizeit höher ist, als in der Allgemeinbevölkerung.
% Es besteht auch die Möglichkeit, dass Personen, die bereits eine Weiterbildung machen
% oder allgemein bereit sind eine Weiterbildung in ihrer Freizeit zu machen, nicht positiv 
% abgestimmt haben

% Schauen, ob die Hypothese abgelehnt oder beibehalten wird