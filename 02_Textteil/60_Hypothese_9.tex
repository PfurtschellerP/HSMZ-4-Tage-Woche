%Hypothese 9: Durch die Einführung der 4-Tage Woche steigt die Bereitschaft zur Weiterbildung.
%V&L

\chapter{Überprüfung der Hypothese 9}
\label{chap:hypothese9}

\section{Vorgehensweise}
Mit der Hypothese \glqq{}Durch die Einführung der 4-Tage Woche steigt die Bereitschaft 
zur Weiterbildung \grqq{} wird ein Fokus auf die 
Auswirkung der 4-Tage-Woche auf die Arbeitnehmenden gelegt. Auf Basis dieser Hypothese 
wurde in der Umfrage
insbesondere die Frage \glqq{}Würde eine 4-Tage-Woche Ihre Bereitschaft zur Weiterbildung in 
Ihrer Freizeit erhöhen? \grqq{} an die Teilnehmenden gestellt.

% Annahme, da die Umfrage in dem näheren Umfeld von Personen durchgeführt wurde, die 
% Berufsbegleitend einen Master machen, kann vermutet werden, dass die Bereitschaft zur
% Weiterbildung in der Freizeit höher ist, als in der Allgemeinbevölkerung.
% Es besteht auch die Möglichkeit, dass Personen, die bereits eine Weiterbildung machen
% oder allgemein bereit sind eine Weiterbildung in ihrer Freizeit zu machen, nicht positiv 
% abgestimmt haben

% Es werden Strukturprüfende Methoden verwendet (Regressionsanalys, Varianzanalyse,
% Diskriminanzanalyse, Kontingenzanalyse)

% Kausalanalys: Wie groß ist der Einfluss der Unabhängigen Variablen auf die Abhängige Variable

% Betrachtung: Ist die Bereitschaft zur Weiterbildung im allgemeinen da
% und besteht ein Zusammenhang zwischen der Bereitschaft zur Weiterbildung und bestimmten Gruppen

% Darstellung der Ergebnisse, ob die Teilnehmenden eher bereit wären für eine Weiterbildung

% Schauen welche Gruppe von Leuten eher bereit sind für eine Weiterbildung
    % Alter - Zusammenhang: Je jünger, desto eher bereit?
    % Geschlecht (Nominal) - Einfache Regression?
    % Kinder (Nominal) - Einfache Regression?
    % Ausschließen von Teilzeit oder nicht?
    % körperliche arbeit? (Ordinal?)
    % Führungsverantwortung - Sind die Personen schon gebildet genug?

    % Auf Statistische Unabhängigkeit achten/prüfen?




\section{Analyse}
% Die in Vorgehensweise erwähnten Analysen durchführen und schauen wie das Ergebnis aussieht

\section{Ergebnis}

% Beantworten, ob die Hypothese abgelehnt oder beibehalten wird