%Hypothese 5: Die Einführung einer 4 Tage Woche führt zu einer Steigerung der 
Attraktivität des Arbeitgebers. 
%P&P

\chapter{Überprüfung der Hypothese 5}
\label{chap:hypothese5}


\section{Vorgehensweise}
Mit der Hypothese \gqq{Die Einführung einer 4-Tage-Woche führt zu einer 
Steigerung der Attraktivität des Arbeitgebers} wird - wie schon in \ref{tab:hyptothesen} beschrieben - der Fokus auf die 
Auswirkung der 4-Tage-Woche auf die Arbeitgebenden gelegt. 

\paragraph{Identifikation der relevanten Variablen}
In einem Ersten Schritt werden die Variablen aus der durchgeführten Umfrage identifiziert, die für die Überprüfung der Hypothese relevant sind.

Als erstes ist dort die Frage \gqq{Könnte die Implementierung einer 4-Tage-Arbeitswoche die Attraktivität Ihres 
Unternehmens / Ihrer Organisation steigern?} zu nennen. Diese dient für die folgende Analyse als abhängige Variable, da sie die Attraktivität des 
Arbeitgebenden in der untersuchten Stichprobe abbildet.

Zu den unabhängigen Variablen zählen in diesem Fall die Fragen, welche die Attraktivität des Arbeitgebenden beeinflussen könnten. Diese sind:

\begin{itemize}
  \item Alter
  \item Geschlecht
  \item Beschäftigungsverhältnis
  \item Branche
  \item (Körperliche Beanspruchung)
  \item Die Position im Unternehmen (Führungskraft oder nicht)
\end{itemize}


\section{Analyse}

\section{Ergebnis}