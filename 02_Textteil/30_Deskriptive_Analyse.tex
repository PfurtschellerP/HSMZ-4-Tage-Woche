\chapter{Deskriptive Analyse}

%Betrachtung was in den Daten enthalten ist (Geschlechter Verteilung, Alter, etc.)
In der durchgeführten wurden zu Beginn allgemeine Daten über die Teilnehmenden erhoben.
Die Informationen welche über die Teilnehmenden abgefragt worden sind, können in der Tabelle
\ref{tab:allgemeine_daten} eingesehen werden.

% \usepackage{float}
\begin{table}[h]
    \centering
    \begin{tabular}{|l|l|p{8cm}|}
        \hline
        \textbf{Variable} & \textbf{Frage} & \textbf{Bedeutung}\\
        \hline
        Geschlecht & QA1 & Männlich, Weiblich, Divers, keine Angabe\\
        Alter & QA2 & In Gruppen von 10 Jahren (Bsw. 16-25, 26-35, 66+)\\
        Kinder & QA3 & Hat die Person Kinder (Ja/Nein)\\
        Jüngstes Kind & QA4 & Wie alt ist das jüngste Kind\\
        Berufstätig & QA5 & Ist die Person Berufstätig (Ja/Nein)\\
        Beschäftigungsverhältnis & QA6 & Ist die Person angestellt oder selbständig tätig\\
        Vollzeit/Teilzeit & QA7 & Ist die Person in Vollzeit oder Teilzeit tätig\\
        Arbeitsstunden & QA8 & Wie viele Stunden arbeitet die Person pro Woche als Teilzeitkraft\\
        Stunden einer Vollzeitwoche & QA9 & Wie viele Stunden hat eine Vollzeitwoche der Person\\
        Schichtmodell & QA10 & Arbeitet die Person im Schichtbetrieb\\
        Körperliche Beanspruchung & QA11 & Wie stark ist die körperliche Belastung der Arbeit\\
        Branche & QA12 & In welcher Branche ist die Person tätig\\
        Anzahl der Mitarbeitenden & QA13 & Wie viele Mitarbeitende hat das Unternehmen\\
        Führungsverantwortung & QA14 & Hat die Person Führungsverantwortung\\
        Teamgröße & QA15 & Wie groß ist das Team unter der Person\\
        Bereits 4TW & QA16 & Hat die Person bereits eine 4-Tage-Woche\\
        \hline
    \end{tabular}
    \caption{Allgemeine Daten der Teilnehmenden}
    \label{tab:allgemeine_daten}
\end{table}

Mit den Informationen kann zum einen eine allgemeine Aussage über die Gruppe der Teilnehmenden 
getroffen werden, als auch zu den Hypothesen im den Kapiteln \ref{chap:hypothese4},
\ref{chap:hypothese5} und \ref{chap:hypothese9}. Dadurch kann in diesem Kapitel auch betrachtet
werden, ob die Stichprobe repräsentativ ist und ob es besondere Ausprägungen in den Daten gibt.
 