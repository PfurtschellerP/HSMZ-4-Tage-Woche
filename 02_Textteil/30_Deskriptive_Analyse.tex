\chapter{Deskriptive Analyse}

%Betrachtung was in den Daten enthalten ist (Geschlechter Verteilung, Alter, etc.)
In der durchgeführten wurden zu Beginn allgemeine Daten über die Teilnehmenden erhoben.
Die Informationen welche über die Teilnehmenden abgefragt worden sind, können in der Tabelle
\ref{tab:allgemeine_daten} eingesehen werden.

% \usepackage{float}
\begin{table}[h]
    \centering
    \begin{tabular}{|l|l|p{8cm}|}
        \hline
        \textbf{Variable} & \textbf{Frage} & \textbf{Bedeutung}\\
        \hline
        Geschlecht & QA1 & Welchem Geschlecht fühlt sich die Person angehörig\\
        Alter & QA2 & In Gruppen von 10 Jahren (Bsw. 16-25, 26-35, 66+)\\
        Kinder & QA3 & Hat die Person Kinder\\
        Vollzeit/Teilzeit & QA7 & Ist die Person in Vollzeit oder Teilzeit tätig\\
        Körperliche Beanspruchung & QA11 & Wie stark ist die körperliche Belastung der Arbeit\\
        Branche & QA12 & In welcher Branche ist die Person tätig\\
        Bereits 4TW & QA16 & Hat die Person bereits eine 4-Tage-Woche\\
        \hline
    \end{tabular}
    \caption{Allgemeine Daten der Teilnehmenden}
    \label{tab:allgemeine_daten}
\end{table}

Mit den Informationen kann zum einen eine allgemeine Aussage über besondere Ausprägungen in der
Gruppe der Teilnehmenden 
getroffen werden, als auch die Hypothesen im den Kapiteln \ref{chap:hypothese4},
\ref{chap:hypothese5} und \ref{chap:hypothese9} mit Bezug auf einzelne Gruppen betrachtet werden. 
Dadurch kann in diesem Kapitel auch betrachtet werden, ob die Stichprobe repräsentativ ist 
und ob es besondere Ausprägungen in den Daten gibt.
 