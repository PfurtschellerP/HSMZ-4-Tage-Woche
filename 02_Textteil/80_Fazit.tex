\chapter{Fazit und Ausblick \textit{\textcolor{gray}{(Autoren: Philipp und Patrick)}}}

Wie bereits im vorangegangen Kapitel festhalten, hätte rückblickend der Fragebogen besser konzipiert werden müssen.
Vor allem die Variablentypen und die Fragestellung hat die Auswahl der anwendbaren Analyseverfahren stark eingeschränkt

Basierend auf den vorliegenden Daten aus der Umfrage, welche im Rahmen dieser Arbeit durchgeführt wurde, 
lässt sich jedoch festhalten, dass die Meinung zur 4-Tage-Woche prinzipiell positiv und von einer Mehrheit
der Befragten befürwortet wird. Dies spiegelt sich auch in international durchgeführten Studien wider,
die ebenfalls eine hohe Zustimmung zur 4-Tage-Woche zeigen \parencite[vgl.][]{4_day_week_limited_4_2023}.

% Bezogen auf die in dieser Arbeit untersuchten Hypothesen, konnte lediglich Hypothese 5 und 9 ansatzweise bestätigt werden.
% Die Einführung einer 4-Tage-Woche steigert somit die Attraktivität eines Arbeitgebenden für einen Großteil der befragten 
% Arbeitnehmenden.
% Und auch die Bereitschaft sich innerhalb der eigenen Freizeit weiterzubilden, scheint von der Einführung
% einer 4-Tage-Woche positiv beeinflusst. Allerdings könnten hier auch andere Faktoren eine Rolle spielen, wie die 
% persönliche Motivation oder das Interesse an Weiterbildung im Allgemeinen. Insbesondere letzteres könnte durch fehlende 
% Fragen im Fragebogen nicht ausreichend abgedeckt worden sein.

% Hypothese 1 konnte nicht bestätigt werden. So zeigt sich, dass die Zustimmung zur 4-Tage-Woche nicht eindeutig 
% von dem Alter der Befragten abhängt. Andere Faktoren, wie das Bedürfnis nach Work-Life-Balance, die Karrierestufe oder 
% persönliche Vorlieben, können eine wichtigere Rolle spielen. Die relativ geringe Stichprobengröße, insbesondere für 
% die älteste Altersgruppe, könnte auch die Verallgemeinerbarkeit dieser Ergebnisse einschränken.

Letztlich konnte aber zumindest eines erzielt werden. Die Autoren dieses Berichts haben sich mit dem Thema 4-Tage-Woche
auseinandergesetzt, konnten einige Einblicke erlangen und haben sich mit dem Konzeptionieren von Umfragen und der 
statistischen Analyse der resultierenden Daten beschäftigt. Dieses Wissen - und vor allem die Erfahrungen im Hinblick auf
potentielle Fehlerquellen - können in zukünftigen Projekten und Arbeiten sicherlich von Nutzen sein.

Spezifisch für kommende Arbeiten zur 4-Tage-Woche lässt sich noch anmerken, dass auch die negativen Aspekte einer
Einführung einer 4-Tage-Woche betrachtet werden sollten, was in dieser Umfrage nicht der Fall war. 
So könnten beispielsweise gefragt werden, ob die Befragten bereit wären, auf einen Teil ihres Gehalts 
zu verzichten, um eine 4-Tage-Woche zu ermöglichen. Auch ob sie bereit wären, länger zu arbeiten, um die gleiche
Vergütung zu erhalten oder ob sie bereit wären, im Alltag Abstriche zu machen um anderen Leuten eine 4-Tage-Woche 
zu ermöglichen. Beispielsweise einen weiteren Tag in der Woche an dem Geschäfte geschlossen sind oder ähnliches.