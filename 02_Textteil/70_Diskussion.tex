\chapter{Diskussion und Reflexion \textit{\textcolor{gray}{(Autoren: Philipp, Subhan und Patrick)}}}

% Da die Antworten auf die Fragen zu den Hypothesen 5 & 9 identisch Aufgebaut gewesen sind und 
% das Skalenniveau der Variablen identisch war, sind die Authoren in der Auswahl der statischen Methoden
% stark eingeschrenkt gewesen. Daher wurden in den Kapiteln \ref{chap:hypothese5} und \ref{chap:hypothese9}
% die gleichen Methoden verwendet.

Die als Teil dieser Arbeit durchgeführte Umfrage stellte die erste bis jetzt von den Autoren durchgeführte
Umfrage in diesem Umfang dar. Es war demnach zu erwarten, dass die Autoren in der Konzeption des Fragebogens
einige Fehler machen würden. 

Rückblickend betrachtet, hätten die Autoren den Fragebogen besser konzipieren müssen.
Vor allem die Variablentypen und die Fragestellung haben die Auswahl der Analyseverfahren stark eingeschränkt. Und schon
vorab den Informationsgehalt der resultierenden Daten unnötiger Weise reduziert.
Außerdem fiel den Autoren während der Auswertung auf, dass es für eine bessere Auswertung sinnvoll gewesen wäre 
Vergleichsproben zu haben. Beispielsweise zu der Bereitschaft zu einer Weiterbildung in der Freizeit. 
Hierbei wäre es interessant gewesen zu wissen, ob die Bereitschaft auch ohne eine 4-Tage-Woche da gewesen 
wäre um dediziert sagen zu können, ob die 4-Tage-Woche einen Einfluss auf die Bereitschaft hat oder nicht.
Es wurde zwar gefragt, ob die Bereitschaft sich erhöhen würde, allerdings weiß man nicht
ob die Bereitschaft bereits ohne eine 4-Tage-Woche vorhanden war und somit sich nicht merklich verändert hätte.

Des Weiteren wäre es sinnvoll gewesen, die Umfrage an einer größeren Stichprobe durchzuführen. Eine größere Stichprobe
hätte es ermöglicht, die Ergebnisse besser zu generalisieren und auch Unterschiede zwischen den Altersgruppen besser
herauszuarbeiten. Auch ein deteilierterer drill-down in den Datensatz insgesamt wäre möglich gewesen.

Abschließend ist es auch notwendig zu erwähnen, dass die Umfrage sich primär auf die Meinung der Befragten konzentriert.
Es ist also nicht gesagt, dass die Befragten auch tatsächlich so handeln würden, wie sie es in der Umfrage angegeben haben.
Der Meinung der Autoren nach ist dies mit ein Grund, warum die Ergebnisse der Umfrage ein nicht wirklich aussagekräftiges Bild 
liefern, wie sich vielleicht erhofft wurde.

% Subhan
% Diskussion und Reflexion
% Die Ergebnisse deuten darauf hin, dass das Alter möglicherweise nicht der einflussreichste Faktor 
% bei der Bestimmung von Meinungen über die 4TW ist. Andere Faktoren, wie Bedürfnisse nach Work-Life-Balance, 
% Karrierestufe oder persönliche Vorlieben, könnten eine wichtigere Rolle spielen. Die relativ geringe 
% Stichprobengröße, insbesondere für die älteste Altersgruppe, könnte auch die Verallgemeinerbarkeit dieser 
% Ergebnisse einschränken.
% Schlussfolgerung
% Während die Daten keine starken Belege für einen linearen Zusammenhang zwischen Alter und Meinung zur 
% 4-Tage-Woche liefern, zeigen sie doch einen möglichen Unterschied in der Meinung zwischen der jüngsten 
% und ältesten Altersgruppe auf. Weitere Untersuchungen mit einer größeren und vielfältigen Stichprobe 
% könnten dazu beitragen, den Zusammenhang zwischen Alter und Einstellungen zur 4TW zu klären.