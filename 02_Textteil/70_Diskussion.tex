\chapter{Diskussion und Reflexion}

% Da die Antworten auf die Fragen zu den Hypothesen 5 & 9 identisch Aufgebaut gewesen sind und 
% das Skalenniveau der Variablen identisch war, sind die Authoren in der Auswahl der statischen Methoden
% stark eingeschrenkt gewesen. Daher wurden in den Kapiteln \ref{chap:hypothese5} und \ref{chap:hypothese9}
% die gleichen Methoden verwendet.

Die als Teil dieser Arbeit durchgeführte Umfrage stellte die erste bis jetzt von den Autoren durchgeführte
Umfrage dar in diesem Umfang dar. Es war demnach zu erwarten, dass die Autoren in der Konzeption des Fragebogens
einige Fehler machen würden. 

Rückblickend betrachtet, hätten die Autoren den Fragebogen besser konzipieren müssen.
Vor allem die Variablentypen und die Fragestellung haben die Auswahl der Analyseverfahren stark eingeschränkt. Und schon
vorab den Informationsgehalt der resultierenden Daten stark eingeschränkt.
Außerdem fiel den Autoren während der Auswertung auf, dass es für eine bessere Auswertung sinnvoll gewesen wäre 
Vergleichsproben zu haben. Beispielsweise zu der Bereitschaft zu einer Weiterbildung in der Freizeit. 
Hierbei wäre es interessant gewesen zu wissen, ob die Bereitschaft auch ohne eine 4-Tage-Woche da gewesen 
wäre oder nicht um dediziert sagen zu können, ob die 4-Tage-Woche einen Einfluss auf die Bereitschaft hat oder nicht.
Es wurde zwar gefragt, ob die Bereitschaft sich erhöhen würde, allerdings weiß man nicht
ob die Bereitschaft bereits ohne eine 4-Tage-Woche vorhanden war und somit sich nicht verändert.

Abschließend ist es auch notwendig zu erwähnen, dass die Umfrage sich primär auf die Meinung der Befragten konzentriert.
Es ist also nicht gesagt, dass die Befragten auch tatsächlich so handeln würden, wie sie es in der Umfrage angegeben haben.