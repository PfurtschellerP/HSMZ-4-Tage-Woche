\chapter{Einleitung \textit{\textcolor{gray}{(Autoren: Philipp und Patrick)}} }

In dieser Arbeit werden die Autoren die 4-Tage-Woche auf Basis einer Umfrage untersuchen. 
Die 4-Tage-Woche ist ein Arbeitszeitmodell, bei dem Arbeitnehmende vier anstatt fünf Tagen pro Woche arbeiten. 
Dabei gibt es im Allgemeinen drei verbreitete Modelle. Im ersten Modell wird lediglich die bestehende Wochenarbeitszeit auf vier anstelle fünf Tage aufgeteilt. 
Im zweiten Modell wird die Wochenarbeitszeit reduziert, jedoch auch der Lohn. Im dritten Modell wird die Wochenarbeitszeit reduziert, der Lohn bleibt allerdings gleich. \parencite[vgl.][]{habdank_deutscher_2024}
Im Rahmen dieser Arbeit wird sich immer auf das letzte Modell bezogen, bei dem die Wochenarbeitszeit reduziert wird, der Lohn jedoch gleich bleibt.

\section{Motivation}

Die traditionelle Arbeitswoche, geprägt von festen Arbeitszeiten und starren Strukturen, steht nicht erst seit kurzem im Fokus von 
Diskussionen über Flexibilisierung und Modernisierung der Arbeitswelt. Die Notwendigkeit, Arbeitszeitmodelle anzupassen, wird immer deutlicher, da sich 
die Anforderungen und Erwartungen von Arbeitnehmenden im Laufe der Zeit verändern. Unter anderem die Corona Pandemie hat diesen Wandel in 
besonderem Maße befeuert. Das Ergebnis macht sich beispielweise in der vermehrten Selbstorganisation unter Arbeitnehmenden und durch den 
durchschlagenden Erfolg des Home-Office bemerkbar. \parencite[vgl.][S. 73]{haide_arbeitswelt_2022}

In einer Zeit, in der Arbeitgebende vermehrt einen Rückgang der während der Corona-Pandemie gewährten Flexibilität anstreben (\cite{elias_googles_2023}; \cite{lee_apple_2022}; \cite{vanian_meta_2023}), 
gewinnt das Thema „New Work“ und besonders Konzepte wie die 4-Tage-Woche wieder an Bedeutung.
Insbesondere die junge Generation - Generation Z - drängt verstärkt auf eine verbesserte Work-Life-Balance und sucht nach Arbeitsmodellen, 
die es ermöglichen, berufliche Verpflichtungen mit persönlichen Interessen und familiären Verpflichtungen in Einklang zu bringen. \parencite[vgl.][S. 10]{onlyfy_wechselwilligkeitsstudie_2023}

\section{Zielsetzung}

Um ein umfassendes Verständnis für die Einstellung zur 4-Tage-Woche und ihre potenziellen Auswirkungen zu gewinnen, 
hat sich das erste Semester des Masterkurs IT-Management der Hochschule Mainz zum Ziel gesetzt, dieses Thema genauer zu untersuchen. 
Durch eine sorgfältige Analyse sollen verschiedene Thesen bezüglich der Einstellung zur verkürzten Arbeitswoche allgemein sowie 
ihrer Auswirkungen auf die Arbeitgeber sowie Arbeitnehmer überprüft werden. Dabei werden verschiedene Aspekte der Arbeitswelt 
betrachtet, von der Produktivität und Effizienz bis hin zur Mitarbeiterzufriedenheit und -bindung. Ziel ist es, die 
vielschichtigen Herausforderungen und Chancen, die mit der Einführung einer 4-Tage-Woche einhergehen, genauer zu identifizieren 
und offenlegen zu können.

\section{Aufbau der Arbeit}

Im folgenden Kapitel wird zunächst die Durchführung der Umfrage und die Datenaufbereitung beschrieben. Darauf aufbauend widmen sich die Autoren der deskriptiven Analyse der erhobenen Daten.
Ziel dessen ist es, in den folgenden Kapiteln die ausgewählten Hypothesen zu überprüfen. Hier haben sich die Autoren dazu entschieden die Hypothesen 1, 5 und 9, aus der Sammlung der zuvor
im Kurs erarbeiteten Hypothesen, zu überprüfen.
Die Überprüfung der Hypothesen erfolgt jeweils getrennt von einander in den Kapiteln \ref{chap:hypothese1}, \ref{chap:hypothese5} und \ref{chap:hypothese9}. Hier werden für die einzelnen Hypothesen
jeweils die Vorgehensweise beschrieben, die Analyse durchgeführt und die Ergebnisse präsentiert.


In einer abschließenden Diskussion werden die Ergebnisse der Hypothesenüberprüfung zusammengeführt, interpretiert und darauf aufbauend ein Fazit gezogen sowie ein Ausblick gegeben.